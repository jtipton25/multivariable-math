\usepackage{booktabs}

\usepackage{calc}
\usepackage{spalign}
\usepackage{jdr-tikz}

\usepackage{tikz}
\usetikzlibrary{backgrounds}
\usetikzlibrary{arrows}
% \usetikzlibrary{snakes}
\usetikzlibrary{decorations}
\usetikzlibrary{matrix}
\usetikzlibrary{positioning,fit}
\usetikzlibrary{shapes,decorations,decorations.pathmorphing}
\usetikzlibrary{decorations.markings}
\usetikzlibrary{calc,arrows.meta,quotes}
% \usetikzlibrary{ipe}
\usetikzlibrary{angles}
\usetikzlibrary{datavisualization.formats.functions}
\usetikzlibrary{math}

\tikzset{
  whitebg/.style={fill=white},
  whitebg nodes/.style={every node/.append style=whitebg,
  every label/.append style=whitebg,
  every pin/.append style=whitebg},
  thin border/.style={inner sep=#1, outer sep=0pt},
  thin border/.default=1pt,
  thin border nodes/.style={every node/.append style={thin border=#1}},
  thin border nodes/.default=1pt,
  every pin edge/.style={<-,thin},
  node is bbox/.style={inner sep=0pt, outer sep=0pt, line width=0pt},
  every matrix/.style={ampersand replacement=\&},
  math matrix/.style={matrix of math nodes,
    left delimiter={(}, right delimiter={)},
    inner sep=0pt, inner ysep=2pt, row sep=.4em, column sep=.6666em,
    nodes={inner sep=0}},
  }

\tikzset{
  vector/.style={very thick, -{Stealth[length=3mm]}},
  thick vector/.style={very thick, -Stealth},
  point/.style={circle, fill=black, inner sep=0pt,
  minimum size=1.5mm, anchor=center},
}